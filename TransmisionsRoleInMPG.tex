% Options for packages loaded elsewhere
\PassOptionsToPackage{unicode}{hyperref}
\PassOptionsToPackage{hyphens}{url}
%
\documentclass[
]{article}
\usepackage{lmodern}
\usepackage{amssymb,amsmath}
\usepackage{ifxetex,ifluatex}
\ifnum 0\ifxetex 1\fi\ifluatex 1\fi=0 % if pdftex
  \usepackage[T1]{fontenc}
  \usepackage[utf8]{inputenc}
  \usepackage{textcomp} % provide euro and other symbols
\else % if luatex or xetex
  \usepackage{unicode-math}
  \defaultfontfeatures{Scale=MatchLowercase}
  \defaultfontfeatures[\rmfamily]{Ligatures=TeX,Scale=1}
\fi
% Use upquote if available, for straight quotes in verbatim environments
\IfFileExists{upquote.sty}{\usepackage{upquote}}{}
\IfFileExists{microtype.sty}{% use microtype if available
  \usepackage[]{microtype}
  \UseMicrotypeSet[protrusion]{basicmath} % disable protrusion for tt fonts
}{}
\makeatletter
\@ifundefined{KOMAClassName}{% if non-KOMA class
  \IfFileExists{parskip.sty}{%
    \usepackage{parskip}
  }{% else
    \setlength{\parindent}{0pt}
    \setlength{\parskip}{6pt plus 2pt minus 1pt}}
}{% if KOMA class
  \KOMAoptions{parskip=half}}
\makeatother
\usepackage{xcolor}
\IfFileExists{xurl.sty}{\usepackage{xurl}}{} % add URL line breaks if available
\IfFileExists{bookmark.sty}{\usepackage{bookmark}}{\usepackage{hyperref}}
\hypersetup{
  pdftitle={Automatic vs Manual Transmisson for better MPG},
  pdfauthor={Sarah Massengill},
  hidelinks,
  pdfcreator={LaTeX via pandoc}}
\urlstyle{same} % disable monospaced font for URLs
\usepackage[margin=1in]{geometry}
\usepackage{color}
\usepackage{fancyvrb}
\newcommand{\VerbBar}{|}
\newcommand{\VERB}{\Verb[commandchars=\\\{\}]}
\DefineVerbatimEnvironment{Highlighting}{Verbatim}{commandchars=\\\{\}}
% Add ',fontsize=\small' for more characters per line
\usepackage{framed}
\definecolor{shadecolor}{RGB}{248,248,248}
\newenvironment{Shaded}{\begin{snugshade}}{\end{snugshade}}
\newcommand{\AlertTok}[1]{\textcolor[rgb]{0.94,0.16,0.16}{#1}}
\newcommand{\AnnotationTok}[1]{\textcolor[rgb]{0.56,0.35,0.01}{\textbf{\textit{#1}}}}
\newcommand{\AttributeTok}[1]{\textcolor[rgb]{0.77,0.63,0.00}{#1}}
\newcommand{\BaseNTok}[1]{\textcolor[rgb]{0.00,0.00,0.81}{#1}}
\newcommand{\BuiltInTok}[1]{#1}
\newcommand{\CharTok}[1]{\textcolor[rgb]{0.31,0.60,0.02}{#1}}
\newcommand{\CommentTok}[1]{\textcolor[rgb]{0.56,0.35,0.01}{\textit{#1}}}
\newcommand{\CommentVarTok}[1]{\textcolor[rgb]{0.56,0.35,0.01}{\textbf{\textit{#1}}}}
\newcommand{\ConstantTok}[1]{\textcolor[rgb]{0.00,0.00,0.00}{#1}}
\newcommand{\ControlFlowTok}[1]{\textcolor[rgb]{0.13,0.29,0.53}{\textbf{#1}}}
\newcommand{\DataTypeTok}[1]{\textcolor[rgb]{0.13,0.29,0.53}{#1}}
\newcommand{\DecValTok}[1]{\textcolor[rgb]{0.00,0.00,0.81}{#1}}
\newcommand{\DocumentationTok}[1]{\textcolor[rgb]{0.56,0.35,0.01}{\textbf{\textit{#1}}}}
\newcommand{\ErrorTok}[1]{\textcolor[rgb]{0.64,0.00,0.00}{\textbf{#1}}}
\newcommand{\ExtensionTok}[1]{#1}
\newcommand{\FloatTok}[1]{\textcolor[rgb]{0.00,0.00,0.81}{#1}}
\newcommand{\FunctionTok}[1]{\textcolor[rgb]{0.00,0.00,0.00}{#1}}
\newcommand{\ImportTok}[1]{#1}
\newcommand{\InformationTok}[1]{\textcolor[rgb]{0.56,0.35,0.01}{\textbf{\textit{#1}}}}
\newcommand{\KeywordTok}[1]{\textcolor[rgb]{0.13,0.29,0.53}{\textbf{#1}}}
\newcommand{\NormalTok}[1]{#1}
\newcommand{\OperatorTok}[1]{\textcolor[rgb]{0.81,0.36,0.00}{\textbf{#1}}}
\newcommand{\OtherTok}[1]{\textcolor[rgb]{0.56,0.35,0.01}{#1}}
\newcommand{\PreprocessorTok}[1]{\textcolor[rgb]{0.56,0.35,0.01}{\textit{#1}}}
\newcommand{\RegionMarkerTok}[1]{#1}
\newcommand{\SpecialCharTok}[1]{\textcolor[rgb]{0.00,0.00,0.00}{#1}}
\newcommand{\SpecialStringTok}[1]{\textcolor[rgb]{0.31,0.60,0.02}{#1}}
\newcommand{\StringTok}[1]{\textcolor[rgb]{0.31,0.60,0.02}{#1}}
\newcommand{\VariableTok}[1]{\textcolor[rgb]{0.00,0.00,0.00}{#1}}
\newcommand{\VerbatimStringTok}[1]{\textcolor[rgb]{0.31,0.60,0.02}{#1}}
\newcommand{\WarningTok}[1]{\textcolor[rgb]{0.56,0.35,0.01}{\textbf{\textit{#1}}}}
\usepackage{graphicx,grffile}
\makeatletter
\def\maxwidth{\ifdim\Gin@nat@width>\linewidth\linewidth\else\Gin@nat@width\fi}
\def\maxheight{\ifdim\Gin@nat@height>\textheight\textheight\else\Gin@nat@height\fi}
\makeatother
% Scale images if necessary, so that they will not overflow the page
% margins by default, and it is still possible to overwrite the defaults
% using explicit options in \includegraphics[width, height, ...]{}
\setkeys{Gin}{width=\maxwidth,height=\maxheight,keepaspectratio}
% Set default figure placement to htbp
\makeatletter
\def\fps@figure{htbp}
\makeatother
\setlength{\emergencystretch}{3em} % prevent overfull lines
\providecommand{\tightlist}{%
  \setlength{\itemsep}{0pt}\setlength{\parskip}{0pt}}
\setcounter{secnumdepth}{-\maxdimen} % remove section numbering

\title{Automatic vs Manual Transmisson for better MPG}
\author{Sarah Massengill}
\date{2/19/2021}

\begin{document}
\maketitle

\hypertarget{executive-summary}{%
\section{Executive Summary}\label{executive-summary}}

``Is an automatic or manual transmission better for MPG''

``Quantify the MPG difference between automatic and manual
transmissions''

\hypertarget{exploratory-analysis-and-model-search}{%
\section{Exploratory Analysis and Model
Search}\label{exploratory-analysis-and-model-search}}

Looking t-test first.\\
H\_0: There is no difference in the mean MPG for an automatic and manual
transmission.\\
H\_A: There is a difference in the mean MPG for an automatic and manual
transmission.

By creating two subgroups in the data, ones who transmission type (am)
is automatic (``auto'') versus manual(``man'') a t.test can be used to
test the null hypothesis. R's t.test function calculates the 95\%
confidence interval of the difference between the two groups' means to
be on average between 3.6 and 10.8 miles per gallon less for vehicles
with automatic transmissions. The differences in the mean is significant
with a p-value equal to 0.00285. The null hypothesis is rejected in
favor of the alternative hypothesis.

Modeling MPG with Transmission Type (am) alone:

Model 1: \(mpg = b_0 + b_1(am)\) \(mpg = 17.1 + 7.2(am)\)

The t-test results are mirrored in the simple linear model that predicts
mpg based on type of transmission (am: auto = 0,man = 1). On average,
switching from an automatic to a manual transmission increases the mean
miles per gallon by 7.2 mpg on average, with a 95\% confidence interval
of {[}3.64,10.84{]}. Since zero is \emph{not} in the interval for
\(b_1\) we could conclude that there is a statistically significant
difference in MPG for automatic and manual transmissions. Unfortunately,
based on the \(R^2\) statistic this linear model only describes about
36\% of the variation in MPG results. If this model described more of
the variation in the results we could stop here, but there are probably
confounding and/or interaction variables that are causing some of the
variance and hiding the true effects of transmission on the MPG of the
vehicles in the data set.

It is interesting to note that this simple linear model appears to
satisfy the assumptions of linearity, constant variance and normally
distributed residuals even if it does not quite answer if an automatic
transmission is better than a manual transmission, but it eludes to
manual transmission being better.

Moving to a multivariable regression:

Does Weight (wt) confound and/or interact the effects of the
Transmission (am) on MPG?

After plotting the MPG with respect to weight (wt) and differentiating
by color the manual and automatic transmissions show a relationship to
weight. The lighter weight vehicles in the data set are mostly manual
while the heavier vehicles are automatic. The weight of the vehicle is
also linearly related with MPG in that the heavier vehicles correlate to
less MPG and vice verse. There are 18 automatic transmission vehicles
over three metric tones and one below, while two manual transmission
vehicle over and 11 under. This split makes it difficult to separate the
effects of each from each other without an interaction term.

\hypertarget{introducing-two-new-multivariable-linear-models}{%
\subsection{Introducing two new multivariable linear
models:}\label{introducing-two-new-multivariable-linear-models}}

\hypertarget{model-2-mpg-b_0-b_1am-b_2wt}{%
\subsubsection{\texorpdfstring{model 2:
\(mpg = b_0 + b_1(am) + b_2(wt)\)}{model 2: mpg = b\_0 + b\_1(am) + b\_2(wt)}}\label{model-2-mpg-b_0-b_1am-b_2wt}}

\hypertarget{model-3-mpg-b_0-b_1am-b_2wt-b_3amwt}{%
\subsubsection{\texorpdfstring{model 3:
\(mpg = b_0 + b_1(am) + b_2(wt)+ b_3(am)(wt)\)}{model 3: mpg = b\_0 + b\_1(am) + b\_2(wt)+ b\_3(am)(wt)}}\label{model-3-mpg-b_0-b_1am-b_2wt-b_3amwt}}

The first of the two models estimates the average effects of
transmission on the mileage while holding the weight constant. The
second model does the same thing but also adds an effect modifier term.

\begin{itemize}
\tightlist
\item
  \(b_0\): average MPG for a automatic transmission (am = ``auto'') when
  the weight of the vehicle (wt) is zero\\
\item
  \(b_1\) average change in MPG when switching from an automatic
  transmission to a manual(am=``man'') controlling for weight (wt =
  constant)\\
\item
  \(b_2\) average change in MPG with a 1-metric ton increase in the
  weight (wt) of the vehicle controlling for transmission type.
\item
  \(b_3\) effect modifier in the second model, this effect modifier will
  alter the effect of the manual transmission on mpg based on the
  weight.
\end{itemize}

Graphically, the first model will have the same slope but the y
intercepts change if \(b_1\) is non-zero, the second model allows for a
different slope and a different intercept between the two lines that
represent the transmission groups.

Results:

model 2:
\(mpg =  \begin{cases}  37.32 -5.4(wt) & \text{if $am = auto$}  \ 37.32 - .02(1) -5.4(wt) & \text{if $am = man$}  \end{cases}\)
model 3:
\(mpg =  \begin{cases}  31.4 -5.4 \cdot wt & \text{if $am = auto$} \\  \ 37.34 + 14.9(1) -5.4\cdot wt & \text{if $am = man$} \\  \end{cases}\)

\hypertarget{exploratory-analysis}{%
\subsection{Exploratory analysis}\label{exploratory-analysis}}

{[}, 1{]} mpg Miles/(US) gallon {[}, 2{]} cyl Number of cylinders {[},
3{]} disp Displacement (cu.in.) {[}, 4{]} hp Gross horsepower {[}, 5{]}
drat Rear axle ratio {[}, 6{]} wt Weight (1000 lbs) {[}, 7{]} qsec 1/4
mile time {[}, 8{]} vs Engine (0 = V-shaped, 1 = straight) {[}, 9{]} am
Transmission (0 = automatic, 1 = manual) {[},10{]} gear Number of
forward gears {[},11{]} carb Number of carburetors

\begin{Shaded}
\begin{Highlighting}[]
\NormalTok{lm.am.wt1<-}\KeywordTok{lm}\NormalTok{(mpg}\OperatorTok{~}\NormalTok{am }\OperatorTok{+}\StringTok{ }\NormalTok{wt,}\DataTypeTok{data =}\NormalTok{ mtcars2)}
\NormalTok{lm.am.wt2<-}\KeywordTok{lm}\NormalTok{(mpg}\OperatorTok{~}\NormalTok{am}\OperatorTok{*}\NormalTok{wt,}\DataTypeTok{data=}\NormalTok{mtcars2)}
\KeywordTok{summary}\NormalTok{(lm.am.wt1)}
\end{Highlighting}
\end{Shaded}

\begin{verbatim}
## 
## Call:
## lm(formula = mpg ~ am + wt, data = mtcars2)
## 
## Residuals:
##     Min      1Q  Median      3Q     Max 
## -4.5295 -2.3619 -0.1317  1.4025  6.8782 
## 
## Coefficients:
##             Estimate Std. Error t value Pr(>|t|)    
## (Intercept) 37.32155    3.05464  12.218 5.84e-13 ***
## amman       -0.02362    1.54565  -0.015    0.988    
## wt          -5.35281    0.78824  -6.791 1.87e-07 ***
## ---
## Signif. codes:  0 '***' 0.001 '**' 0.01 '*' 0.05 '.' 0.1 ' ' 1
## 
## Residual standard error: 3.098 on 29 degrees of freedom
## Multiple R-squared:  0.7528, Adjusted R-squared:  0.7358 
## F-statistic: 44.17 on 2 and 29 DF,  p-value: 1.579e-09
\end{verbatim}

\begin{Shaded}
\begin{Highlighting}[]
\NormalTok{intm1}\FloatTok{.1}\NormalTok{ =}\StringTok{ }\KeywordTok{coef}\NormalTok{(lm.am.wt1)[}\DecValTok{1}\NormalTok{]}
\NormalTok{slopem1=}\KeywordTok{coef}\NormalTok{(lm.am.wt1)[}\DecValTok{3}\NormalTok{]}
\NormalTok{intm1}\FloatTok{.2}\NormalTok{ =}\StringTok{ }\KeywordTok{sum}\NormalTok{(}\KeywordTok{coef}\NormalTok{(lm.am.wt1)[}\DecValTok{1}\OperatorTok{:}\DecValTok{3}\NormalTok{])}
\KeywordTok{summary}\NormalTok{(lm.am.wt2)}
\end{Highlighting}
\end{Shaded}

\begin{verbatim}
## 
## Call:
## lm(formula = mpg ~ am * wt, data = mtcars2)
## 
## Residuals:
##     Min      1Q  Median      3Q     Max 
## -3.6004 -1.5446 -0.5325  0.9012  6.0909 
## 
## Coefficients:
##             Estimate Std. Error t value Pr(>|t|)    
## (Intercept)  31.4161     3.0201  10.402 4.00e-11 ***
## amman        14.8784     4.2640   3.489  0.00162 ** 
## wt           -3.7859     0.7856  -4.819 4.55e-05 ***
## amman:wt     -5.2984     1.4447  -3.667  0.00102 ** 
## ---
## Signif. codes:  0 '***' 0.001 '**' 0.01 '*' 0.05 '.' 0.1 ' ' 1
## 
## Residual standard error: 2.591 on 28 degrees of freedom
## Multiple R-squared:  0.833,  Adjusted R-squared:  0.8151 
## F-statistic: 46.57 on 3 and 28 DF,  p-value: 5.209e-11
\end{verbatim}

\begin{Shaded}
\begin{Highlighting}[]
\CommentTok{# M<-cor(mtcars)}
\CommentTok{# corrplot(M, method="number")}
\CommentTok{## plot}
\KeywordTok{table}\NormalTok{(mtcars2}\OperatorTok{$}\NormalTok{am,mtcars2}\OperatorTok{$}\NormalTok{wt.cat)}
\end{Highlighting}
\end{Shaded}

\begin{verbatim}
##       
##        [0,3) [3,6)
##   auto     1    18
##   man     11     2
\end{verbatim}

\hypertarget{appendix}{%
\subsection{Appendix}\label{appendix}}

\hypertarget{model-1-plots}{%
\subsubsection{Model 1 Plots}\label{model-1-plots}}

\includegraphics{TransmisionsRoleInMPG_files/figure-latex/plotsm1-1.pdf}
\includegraphics{TransmisionsRoleInMPG_files/figure-latex/plotsm1-2.pdf}

\hypertarget{model-2-3-plots}{%
\subsubsection{Model 2 \& 3 plots}\label{model-2-3-plots}}

\includegraphics{TransmisionsRoleInMPG_files/figure-latex/m2-1.pdf}
\includegraphics{TransmisionsRoleInMPG_files/figure-latex/m2-2.pdf} \#\#
Including Plots

You can also embed plots, for example:

\includegraphics{TransmisionsRoleInMPG_files/figure-latex/pressure-1.pdf}

Note that the \texttt{echo\ =\ FALSE} parameter was added to the code
chunk to prevent printing of the R code that generated the plot.

\end{document}
